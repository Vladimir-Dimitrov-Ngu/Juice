\documentclass[unicode]{beamer}
\usepackage{cmap}
\usepackage[utf8]{inputenc}
\usepackage[T2A]{fontenc}
\usepackage[russian]{babel}
\usepackage{amsthm, amsmath, amssymb}
\usetheme{Madrid} % theme in latex
\usepackage{hyperref}
\definecolor{linkcolor}{HTML}{799B03} % цвет ссылок
\definecolor{urlcolor}{HTML}{799B03} % цвет гиперссылок

\hypersetup{pdfstartview=FitH,  linkcolor=linkcolor,urlcolor=urlcolor, colorlinks=false}
%\usepackage{amsmath}

\title[Анализ спроса]{Анализ спроса на рынке соков в Перми}
%\subtitle[]
\author{Димитров В.К}
\institute[НГУ]{Новосибирский Государственный университет}
\logo{\includegraphics[width=1cm]{Картинки/znak_2.pdf}}
\begin{document}
	
\begin{frame}{Название слайда}
	\titlepage


\end{frame}

	\begin{frame}
	\frametitle{Содержание}
	
	\tableofcontents [pausesections]
	
	
\end{frame}


\section{Постановка задачи}
\begin{frame}
	\frametitle{\insertsection}
	
	\begin{enumerate}
		\item 
		Задача: предсказание спроса на рынке соков в Перми.
		\item 
		Количество признаков 25  (после обработки 40).
		\item
		Целевая переменная - количество продаж.
		
		\item 
		Методы: 
		\begin{enumerate}
			\item Эконометричкский подход
			
			\item Инженерный подход
			
			\item Смешенный подход
		\end{enumerate}
	\end{enumerate}
\end{frame}
\section{Эконометрический подход}
\subtitle{Пуассоновская регрессия}

\begin{frame}
	\frametitle{\insertsection}
	
	\framesubtitle{\insertsubtitle}
	
	\begin{block}{Распределение Пуассона}
		
		$\xi$ - это случайная величина, которая имеет распределение Пуассона. Тогда		
		
		
		\[P\{\xi = k\} = \dfrac{\exp(-\lambda)\times \lambda^k}{k!}, k = \{0, 1, \ldots \}\]
		
	\end{block}	
\end{frame}

\begin{frame}
	\frametitle{\insertsubtitle}
	\begin{block}{Эконометрическая спецификация}
		Эконометрическая специфкаиция предложениия , заключается в :
		
		\[
		\lambda = \exp(x'\beta)
		\]

	\end{block}\pause


	\begin{block}{Модель для $y_i$ покупки}
		Пусть $y_i = k$, тогда наша модель перепишется в виде:
		
	\[
	P(\xi = y_i) = \dfrac{\exp(\exp(x' * \beta)) \times \exp(x'\beta)^{y_{i}}}{y_i!}
	\]
		
		
	\end{block}
\end{frame}

\begin{frame}
	\frametitle{\insertsubtitle}
	\begin{block}{Максимум правдоподобия}
		
		Для оценки используется оценка максимального правдоподобия. Тогда задача переписывается в виде:

		$$\max\limits_{{\beta}}\dfrac{1}{N}\sum_{i = 1}^{N}\{-\exp(x_{i}'\beta) + y_{i}x_{i}'\beta - \ln(y_i)!\}$$

	\end{block}

	\begin{block}{Но...}
		
		 Задача не имеет явного решения
		
		
	\end{block}
\end{frame}
\begin{frame}
	\frametitle{Итеративные методы}
	\begin{enumerate}
		\item Поиск по сетке (Grid search)
		
		\item Градиентный спуск (Gradient method)
		
		\item Метод Ньютона-Рафсона (Newton–Raphson Method)
		
	\end{enumerate}
\end{frame}
\begin{frame}{GD}
		\begin{block}{Градиентый метод}
			\textbf{Градиентные методы (Gradient Method)}. Новые значения признаков пересчитываются по формуле:
			$$\hat{\beta}_{i+1} = \hat{\beta}_{i} + A_{i}g_{i} $$ 
		\end{block}

\end{frame}
\begin{frame}
	\frametitle{Численные методы (Пример)}
	
	\centering
		\includegraphics[scale=1]{Картинки/picture_2.png}

\end{frame}

\begin{frame}
	\frametitle{Метод НР}
	\begin{block}{Описание метода}
		Так как наша функция глобально вогнутая, тогда метод НР работает достаточно хорошо. Градиентный спуск модифицируется:
		
		
		$$\hat{\beta_{i+1}} = \hat{\beta_{i}} - H^{-1}_{i} g_{i} $$ \pause
		
		
		
		$H_i =  \dfrac{\partial^2 Q_N(\beta)}{\partial \beta \partial \beta'}\bigg|_{\hat{\beta}_i}$ - гессиан функции $Q(\beta)$ в точке $\hat{\beta}$.
		
	\end{block}
	
\end{frame}
\begin{frame}{Метод НР}
\begin{alertblock}{Проблемы}		
	\begin{enumerate}
		\item Как оптимизировать Гессиан? \pause
		
		\item Как выбирать подвыборку? \pause
		
		\item Сколько итераций и иниицализация? \pause
		
		\item Learning rate \pause
		
		\item Написать программу 
	\end{enumerate}
\end{alertblock}
\end{frame}

\begin{frame}
	\frametitle{Дополнение}
	\begin{exampleblock}{Расширение спецификации}
		\large
		\begin{itemize}
			
			\item Пуассон с раздутым нулем (zero inflated poisson)
			
			\item Мультиномиальная логистическая модель
			
			\item Специфифация для параметра $\lambda$
		\end{itemize}
	\end{exampleblock}
	
\end{frame}
\section{Машинное обучение}
\begin{frame}{Машинное обученение}
	\begin{itemize}
		\Large
		\item  Логистическая регрессия
		
		\item  Random Forest
		
		\item GradientBoostingClassifier
	\end{itemize}
	
\end{frame}
\begin{frame}{Точность}
	\centering
	\Large
	\begin{tabular}{ | l | l | l | }
		\hline
		Модель & Параметры & Точность(до, после) \\ \hline
		Логистическая  & lr, alpha & 0.74, 0.76 \\
		Random Forest &  Много & 0.72, 0.77\\
		GBR &  Много & 0.84 \\
		\hline
	\end{tabular}


	
\end{frame}

\begin{frame}{Ссылка на ноутбук}
	
	\href{https://github.com/Harrix/Math-Harrix-Library}{Мой notebook}
	
\end{frame}
\section{Смешанная модель}

\begin{frame}
	\frametitle{\insertsection}
		\begin{itemize}
			\item Можно взять предсказание эконометрической модели и подать на вход любой модели ML (stacking).
			
			\item Можно использовать методы кластеризации, для выявление новых признаков для эконометричских моделей.
		\end{itemize}
\end{frame}	
	
\begin{frame}
	\frametitle{Пример кластеризации}
	\centering
	\includegraphics{Картинки/picture_3.png}
	
	
\end{frame}
	


\section{Вопросы}
\begin{frame}
	\frametitle{Вопросы}
	\begin{block}{Научный блок}
		\begin{itemize}
			\item Эластичность
			
			\item Интепретация коэффициентов
			
			\item Индексы
			
			\item Получение состоятельных оценок
		\end{itemize}

		
	\end{block} \pause

	\begin{exampleblock}{Бизнес блок}
		\begin{itemize}
			\item Как перевести метрику качества в метрику бизнеса ?
			
			\item Что может быть интересно бизнесу?
		\end{itemize}
	\end{exampleblock}
\end{frame}

\begin{frame}{Вопросы}
	\centering
	\huge
	Готов ответить на ваши вопросы.
\end{frame}
%Заключительный слайд
\begin{frame}{Благодарность}
	\centering
	\huge
	Спасибо за внимание!
	
\end{frame}



\end{document}
